% ==============================================================================
% QUICK REFERENCE: Figure Inclusion for geometric_atom_v2.tex
% ==============================================================================
%
% This file shows how to insert the generated figures into your LaTeX paper.
% Simply copy-paste the relevant sections below into your .tex file.
%
% ==============================================================================

% ------------------------------------------------------------------------------
% FIGURE 1: 3D Paraboloid Architecture (Single Column)
% ------------------------------------------------------------------------------
% Insert at the end of Section 2 or beginning of Section 3

\begin{figure}[htbp]
\centering
\includegraphics[width=0.95\linewidth]{figure1_paraboloid_3d.pdf}
\caption{The discrete paraboloid lattice for $n_{\text{max}}=5$ (55 quantum 
states). Nodes represent states $\ket{n,l,m}$ colored by principal quantum 
number $n$ (viridis colormap). Grey lines denote angular momentum connections 
($L_\pm$) forming concentric rings at fixed energy. Red lines show radial 
transitions ($T_\pm$) connecting adjacent shells. The paraboloid geometry 
simultaneously encodes spatial extent ($r = n^2$) and binding energy 
($z = -1/n^2$), making quantum state space geometrically visible.}
\label{fig:paraboloid3d}
\end{figure}

% Reference in text:
% "Figure~\ref{fig:paraboloid3d} shows the 3D structure..."


% ------------------------------------------------------------------------------
% FIGURE 2: Transition Pathways (Two Column / Full Width)
% ------------------------------------------------------------------------------
% Insert in Section 4 or 5 for impact

\begin{figure*}[htbp]
\centering
\includegraphics[width=\linewidth]{figure2_transition_path.pdf}
\caption{Visualization of quantum transitions on the paraboloid lattice. 
\textbf{Left panel:} Side view (radial distance vs.\ energy coordinate) 
showing a Balmer series transition $\ket{3,1,0} \to \ket{2,1,0}$. The red 
arrow illustrates photon emission as geometric descent in the energy well. 
\textbf{Right panel:} Top view (x-y projection) revealing the $SO(4)$ angular 
structure. States on the $n=2$ shell (red circle) serve as final states for 
visible spectral lines. Grey lines show the dense web of angular connections 
preserving $n$.}
\label{fig:transitions}
\end{figure*}

% Reference in text:
% "As shown in Fig.~\ref{fig:transitions}, quantum transitions..."
% "The left panel of Fig.~\ref{fig:transitions} demonstrates..."


% ------------------------------------------------------------------------------
% FIGURE 3: Sparsity Structure (Two Column / Full Width)
% ------------------------------------------------------------------------------
% Insert in Section 4 (Computational Verification)

\begin{figure*}[htbp]
\centering
\includegraphics[width=\linewidth]{figure3_sparsity.pdf}
\caption{Sparse matrix structure of key operators for $n_{\text{max}}=7$ 
(140 states, $140 \times 140$ matrices). Non-zero elements appear as colored 
pixels. \textbf{Top left:} Radial raising operator $T_+$ (red) with 70 
non-zero elements (0.36\% density). \textbf{Top right:} Angular raising 
operator $L_+$ (blue) with 84 elements (0.43\%). \textbf{Bottom left:} 
Angular momentum Casimir $L^2$ (green) showing block-diagonal structure by 
$l$. \textbf{Bottom right:} Approximate Hamiltonian $H \approx T_3 + 0.1 L^2$ 
(purple) demonstrating combined sparsity. Despite the complex 3D geometry, 
operators remain $\sim 1\%$ dense, enabling $O(n)$ computational scaling. 
The inset shows total memory compression exceeds 99\%.}
\label{fig:sparsity}
\end{figure*}

% Reference in text:
% "Figure~\ref{fig:sparsity} demonstrates the sparse structure..."
% "The spy plots in Fig.~\ref{fig:sparsity} reveal..."


% ==============================================================================
% OPTIONAL: All Three Figures in Appendix (For Long Papers)
% ==============================================================================

\appendix
\section{Lattice Visualization}

\begin{figure}[h]
\centering
\includegraphics[width=0.85\linewidth]{figure1_paraboloid_3d.pdf}
\caption{3D paraboloid lattice structure. See main text for details.}
\label{fig:paraboloid3d_appendix}
\end{figure}

\begin{figure}[h]
\centering
\includegraphics[width=0.85\linewidth]{figure2_transition_path.pdf}
\caption{Transition pathways. See main text for details.}
\label{fig:transitions_appendix}
\end{figure}

\begin{figure}[h]
\centering
\includegraphics[width=0.85\linewidth]{figure3_sparsity.pdf}
\caption{Operator sparsity structure. See main text for details.}
\label{fig:sparsity_appendix}
\end{figure}


% ==============================================================================
% TIPS AND TRICKS
% ==============================================================================

% TIP 1: Adjusting Figure Placement
% Replace [htbp] with:
%   [h]  - Try to place here
%   [t]  - Top of page
%   [b]  - Bottom of page
%   [p]  - Page of floats only
%   [H]  - Force exactly here (requires \usepackage{float})

% TIP 2: Side-by-Side Figures (Alternative to figure*)
\begin{figure}[htbp]
\centering
\begin{minipage}{0.48\linewidth}
    \includegraphics[width=\linewidth]{figure1_paraboloid_3d.pdf}
    \caption{Part A}
\end{minipage}
\hfill
\begin{minipage}{0.48\linewidth}
    \includegraphics[width=\linewidth]{figure2_transition_path.pdf}
    \caption{Part B}
\end{minipage}
\end{figure}

% TIP 3: Subfigures (Requires \usepackage{subcaption})
\begin{figure}[htbp]
\centering
\begin{subfigure}{0.48\linewidth}
    \includegraphics[width=\linewidth]{figure1_paraboloid_3d.pdf}
    \caption{3D view}
    \label{fig:sub1}
\end{subfigure}
\hfill
\begin{subfigure}{0.48\linewidth}
    \includegraphics[width=\linewidth]{figure2_transition_path.pdf}
    \caption{2D projection}
    \label{fig:sub2}
\end{subfigure}
\caption{Combined figure with subfigures.}
\label{fig:combined}
\end{figure}
% Reference: Fig.~\ref{fig:combined}\subref{fig:sub1}

% TIP 4: Wrapping Text Around Figure (Requires \usepackage{wrapfig})
\begin{wrapfigure}{r}{0.5\linewidth}
    \centering
    \includegraphics[width=0.95\linewidth]{figure1_paraboloid_3d.pdf}
    \caption{Wrapped figure}
    \label{fig:wrapped}
\end{wrapfigure}

% TIP 5: Rotating Figure (Requires \usepackage{rotating})
\begin{sidewaysfigure}
    \centering
    \includegraphics[width=0.9\linewidth]{figure3_sparsity.pdf}
    \caption{Landscape figure}
    \label{fig:landscape}
\end{sidewaysfigure}

% TIP 6: High-Resolution for Print
% The generated PDFs are already vector graphics (infinite resolution)
% If journal requires TIFF/EPS:
%   - Convert using: convert -density 600 figure1.pdf figure1.tiff
%   - Or use: pdftops -eps figure1.pdf figure1.eps

% TIP 7: Colorblind-Friendly Check
% The viridis colormap used in figures is colorblind-safe
% Test at: https://www.color-blindness.com/coblis-color-blindness-simulator/

% TIP 8: Figure Size Guidelines by Journal
% Physical Review: \columnwidth or \textwidth
% Nature/Science: Usually 89mm (single) or 183mm (double) width
% Elsevier: 90mm (single) or 190mm (double) width

% ==============================================================================
% END OF QUICK REFERENCE
% ==============================================================================
